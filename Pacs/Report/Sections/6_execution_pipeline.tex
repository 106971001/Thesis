\section{Execution Pipeline}
\label{sec:execution_pipeline}

In this section we describe the full pipeline of the program, which is schematically represented in Figure \ref{sec:execution_pipeline}. This pipeline allows to run the learning algorithm for the asset allocation problem and automatically determine a trading strategy. The execution consists of the following steps. 

\paragraph{Compilation} To build the \lstinline{thesis} library it is sufficient to run the \lstinline{Makefile} generated with \lstinline{cmake}. This produces two executables: \lstinline{main} which is used to debug the program in the \lstinline{Code::Blocks} IDE and \lstinline{main_thesis} which is used to run the experiment in the full execution pipeline. 

\paragraph{\lstinline{generate_synthetic_series.py}} This Python script simulates the returns of a synthetic asset and prints them on a \lstinline{.csv} file which is then read by \lstinline{main_thesis} and used to initialize the \lstinline{MarketEnvironment} object. Alternatively, \lstinline{market_data_collector.py} collects the historical returns for a list of given assets from Yahoo finance.

\paragraph{\lstinline{experiment_launcher.py}} This Python script manages the execution pipeline. First, the experiment parameters are specified and dumped in a \lstinline{.pot} file which is then read by \lstinline{main_thesis}. Given the parameter values, the script determines the folders where the output should be written so that the results can be easily associated to a specific set of parameters. Subsequently, it launches the \lstinline{main_thesis} executable passing the correct parameters via the command line. Finally it runs the \lstinline{postprocessing.py} scripts which processes the output files.  

\paragraph{\lstinline{main_thesis}} This executable takes some inputs from the command line, such as the algorithm to use, the paths to the input files and the paths where the output files should be generated. The experiment parameters are then read from the \lstinline{.pot} file generated by the \lstinline{experiment_launcher.py} using \lstinline{GetPot}. The type of learning algorithm is specified by a string passed to the executable via command line and then used by the factory \lstinline{FactoryOfAgents} to instantiate the corresponding \lstinline{Agent}. When the \lstinline{AssetAllocationExperiment} is run, it outputs various statistics to the given destination folders. More in detail, it prints two files for every independent run of the experiment: a \lstinline{debug.csv} file which contains the learning curves of the algorithm and an \lstinline{output.csv} file which contains the backtest performance measures for the trading strategy learned by the \lstinline{Agent} during training.   

\paragraph{\lstinline{postprocessing.py}} This Python script processes the various files produced by \lstinline{main_thesis} and generates an aggregate analysis of the various learning algorithms, so that they can be easily compared and assessed. In particular, it computes the average and confidence intervals for the learning curves of the algorithms and the backtest cumulative profits of the learned strategies. Moreover, it computes some performance measures typically used in Finance to evaluate a trading strategy, such as the Sharpe ratio and the maximum drawdown. The results of this analysis are stored in some \lstinline{.csv} files and some images are generated using the Python library \lstinline{matplotlib}. 


\begin{sidewaysfigure}[t!]
	\centering
	\begin{tikzpicture}[node distance = 6em, auto, thick]
		\node (rect) at (-9.5,-2) [draw,thick,minimum width=3cm,minimum height=3cm] (generate_synthetic_series) {};
		\node (rect) at (0,0) [draw,thick,minimum width=15cm,minimum height=10cm] (experiment_launcher) {};
		\node (rect) at (+9,-2) [red,draw,thick,minimum width=2cm,minimum height=3cm] (convergence) {};
		\node (rect) at (+9,+2) [red,draw,thick,minimum width=2cm,minimum height=3cm] (performance) {};
		\node (rect) at (-6,2) [red,draw,thick,minimum width=2cm,minimum height=3cm] (input) {};
		\node (rect) at (-6,-2) [red,draw,thick,minimum width=2cm,minimum height=3cm] (synthetic) {};
		\node (rect) at (-2,0) [blue, draw,thick,minimum width=4cm,minimum height=8cm] (main_thesis) {};
		\node (rect) at (2,2) [red,draw,thick,minimum width=2cm,minimum height=3cm] (output) {};
		\node (rect) at (2,-2) [red,draw,thick,minimum width=2cm,minimum height=3cm] (debug) {};
		\node (rect) at (5.5,0) [draw,thick,minimum width=3cm,minimum height=3cm] (postprocessing) {};
		
		\draw (0,5.5) node {\lstinline{experiment_launcher.py}};
		\draw (-9.5,0) node {\lstinline{generate_synthetic_series.py}};
		\draw (-6,-4) node {\lstinline{synthetic.csv}};
		\draw (-7.8,4) node {\lstinline{Single_Synth_RN_P0_F0_S0_N5.pot}};	
		\draw (5.5,2) node {\lstinline{postprocessing.py}};
		\draw (2,4) node {\lstinline{output.csv}};
		\draw (2,-4) node {\lstinline{debug.csv}};
		\draw (9.5,4) node {\lstinline{performance.csv}};
		\draw (9.5,-4) node {\lstinline{convergence.csv}};
		\draw (-2,0) node {\lstinline{main_thesis}};	
	
		\draw[line] (generate_synthetic_series.0) -- (synthetic.180);
		\draw[line] (debug.0) -- (postprocessing.180);
		\draw[line] (output.0) -- (postprocessing.180);
		\draw[line] (postprocessing.0) -- (performance.180);
		\draw[line] (postprocessing.0) -- (convergence.180);
		\draw[line] (input.0) -- (main_thesis.135);
		\draw[line] (synthetic.0) -- (main_thesis.225);
		\draw[line] (main_thesis.45) -- (output.180);
		\draw[line] (main_thesis.315) -- (debug.180);
	\end{tikzpicture}
	\caption{Execution flow of an asset allocation experiment. Black boxes denote  Python scripts, blue boxes executables while red boxes input/output files.}
	\label{fig:pybrain}
\end{sidewaysfigure}

\clearpage