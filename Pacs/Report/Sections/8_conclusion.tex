\section{Conclusion}
\label{sec:conclusion}

Reinforcement learning is a general class of algorithms that allow an agent to learn how to behave in a stochastic and possibly unknown environment simply by trial-and-error. In this paper, after recalling the basic concepts of RL, we presented an application of some state-of-the-art learning algorithms to the classical financial problem of determining a profitable long-short trading strategy. For a synthetic asset, the strategies produced outperform the simple Buy \& Hold strategy, even when investment decisions are only based on extremely basic autoregressive features. Contrarily to standard prediction-based trading systems, the learning algorithms employed are able to adapt as expected to the introduction of transaction costs by reducing the frequency of reallocation and of short positions. This shows the potential of RL techniques in effectively dealing with complex sequential decision problems that are typical in financial applications.\\
The algorithms above have also been tested on historical data but we couldn't find a trading strategy that consistently outperformed the simple Buy \& Hold strategy. This was partly expected as it is extremely difficult to find profitable daily patterns in highly liquid assets, especially when using such simple features. We believe that developing more complex features for the trading strategy, perhaps employing some deep learning feature extraction techniques such as deep auto-encoders or deep (recurrent) neural networks, would allow us to beat the market. Another possibility we considered is to decrease the sampling frequency of the historical data and let the algorithms look for some profitable patters. At the moment, these ideas are still work in progress.  