\chapter{Financial Applications of Reinforcement Learning}

\section{Bibliographical Survey}



\section{Asset Allocation}

The asset allocation problem consists of determining how to dynamically invest
the available capital in a portfolio of different assets in order to maximize
the expected total return or another relevant performance measure. Let us
consider a financial market consisting of $I+1$ different stocks that are
traded only at discrete times $t \in \{0, 1, 2, \ldots\}$ and denote by
${Z}_t = {(Z_t^0, Z_t^1, \ldots, Z_t^I)}^T$ their prices at time $t$.
Typically, $Z_t^0$ refers to a riskless asset whose dynamic is given by $Z_t^0
= {(1 + X)}^t$ where $X$ is the deterministic risk-free interest rate. The
investment process works as follows: at time $t$, the investor observes the
state of the market $S_t$, consisting for example of the past asset prices and
other relevant economic variables, and subsequently chooses how to rebalance
his portfolio, by specifying the units of each stock ${n}_t = {(n_t^0 ,
n_t^1 , \ldots , n_t^I)}^T$ to be held between $t$ and $t+1$. In doing so, he
needs to take into account the transaction costs that he has to pay to the
broker to change his position.  At time $t+1$, the investor realizes a profit
or a loss from his investment due to the stochastic variation of the stock
values. The investor’s goal is to maximize a given performance measure.

\subsection{Reward Function}
Let $W_t$ denote the wealth of the investor at time $t$. The profit realized
between $t$ and $t+1$ is simply given by the difference between the trading
results and the transaction costs payed to the broker. More formally
\begin{equation*}
	\Delta W_{t+1} = W_{t+1} - W_t = \text{PNL}_{t+1} - \text{TC}_{t}	
\end{equation*}
where $\text{PNL}_{t+1}$ denotes the profit due to the variation of the
portfolio asset prices between $t$ and $t+1$
\begin{equation*}
	\text{PNL}_{t+1} = {n}_t \cdot \Delta{Z}_{t+1} = \sum^{I}_{i=0} 
	n_t^i (Z_{t+1}^i - Z_t^i) 
\end{equation*}
and $\text{TC}_t$ denotes the fees payed to the broker to change the portfolio
allocation and on the short positions
\begin{equation*}
	\text{TC}_t = \sum^{I}_{i=0} \delta_p^i \left| n_t^i - n_{t-1}^i\right| Z_t^i 
				- \delta_f W_t \ind{{n}_t \neq {n}_{t-1}} 
				- \sum^{I}_{i=0} \delta_s^i {(n_t^i)}^{-} Z_t^i
\end{equation*}
The transaction costs consist of three different components. The first term 
represent a transaction cost that is proportional to the change in value of the 
position in each asset. The second term is a fixed fraction of the total value
of the portfolio which is payed only if the allocation is changed. The last
term represents the fees payed to the broker for the shares borrowed to build a
short position. The portfolio return between $t$ and $t+1$ is thus given by
\begin{equation}\label{eq:portfolio_return}
	X_{t+1} = \frac{\Delta W_{t+1}}{W_t} = \sum^{I}_{i=0} \left[ a_t^i
	X_{t+1}^i - \delta_i \left| a_t^i - \tilde{a}_t^i \right| - \delta_s
	{(a_t^i)}^- \right] - \delta_f \ind{{a}_t \neq \tilde{{a}}_{t-1}}  
\end{equation}
where 
\begin{equation*}
	X_{t+1}^i = \frac{\Delta Z_{t+1}^i}{Z_t^i}
\end{equation*}
is the return of the $i$-th stock between $t$ and $t+1$, 
\begin{equation*}
	a_t^i = \frac{n_t^i Z_t^i}{W_t}
\end{equation*}
is the fraction of wealth invested in the $i$-th stock between time $t$ and
$t+1$ and finally 
\begin{equation*}
	\tilde{a}_t^i = \frac{n_{t-1}^i Z_t^i}{W_t} = \frac{a_{t-1}^i (1+X_t^i)}
	{1 + X_t}
\end{equation*}
is the fraction of wealth invested in the $i$-th stock just before the 
reallocation. We assume that the agent invests all his wealth at each step, so 
that $W_t$ can be also interpreted as the value of his portfolio. This 
assumption leads to the following constraint on the portfolio weights
\begin{equation}
	\sum^{I}_{i=0} a_t^i = 1 \;\;\;\;\; \forall t \in \{0, 1, 2, \ldots\}
\end{equation}
We notice that we are neglecting the typical margin requirements on the short
positions, which would reduce the available capital at time $t$. Considering
margin requirements would lead to a more complex constraint on the portfolio
weights which would be difficult to treat in the reinforcement learning
framework. Plugging this constraint into Eq. (\ref{eq:portfolio_return}), we
obtain
\begin{equation}\label{eq:portfolio_return_benchmark}
	X_{t+1} = X + \sum^{I}_{i=1} a_t^i (X_{t+1}^i - X) - \sum^{I}_{i=0}
	\left[\delta_i \left| a_t^i - \tilde{a}_t^i \right| - \delta_s^i
	{(a_t^i)}^-\right] - \delta_f \ind{{a}_t \neq \tilde{{a}}_{t-1}}   
\end{equation}
which highlights the role of the risk-free asset as a benchmark for the 
portfolio returns. The total profit realized by the investor between $t=0$ and
$T$ is 
\begin{equation*}
	\Pi_T = W_T - W_0 = \sum^{T}_{t=1} \Delta W_t = \sum^{T}_{t=1} W_t X_t  
\end{equation*}
The portfolio return between $t=0$ and $T$ is given by
\begin{equation*}
	X_{0,T} = \frac{W_T}{W_0} - 1 = \prod_{t=1}^T (1+X_t) - 1
\end{equation*}
In order to cast the asset allocation problem in the reinforcement learning
framework, we consider the log-return of the portfolio between $t=0$ and $T$
\begin{equation}
	R_{0,T} = \log \frac{W_T}{W_0} = \sum^{T}_{t=1} \log(1+X_t) = \sum_{t=1}^T
	R_t
\end{equation}
where $R_{t+1}$ is the log-return of the portfolio between $t$ and $t+1$
\begin{equation}
	R_{t+1} = \log \left\{ 1 + \sum^{I}_{i=0} \left[ a_t^i X_{t+1}^i - \delta_i
	\left| a_t^i - \tilde{a}_t^i \right| - \delta_s {(a_t^i)}^- \right] -
	\delta_f \ind{{a}_t \neq \tilde{{a}}_{t-1}}\right\}
\end{equation}
The portfolio return and log-return can be used as the reward function of a
RL algorithm, either in a offline or in an online approach.

\subsection{States}
At each time step, the agent observes the state of the system to  make his decisions. First, we consider the $P+1$ past returns of all risky assets, i.e. $\{X_t, X_{t-1}, \ldots, X_{t-P}\}$. These input variables may be used to construct more complex features for example using some deep learning techniques, such as a deep auto-encoder. In order to properly incorporate the effects of transaction costs into his decision process, the agent must keep track of its current position $\tilde{a}_t$. Finally, we might consider some external variables $Y_t$ that may be relevant to the trader, such as the common technical indicator used in practice. Summing up, we assume that the state of the system is composed in the following way 
\begin{equation}
	S_t = \{X_t, X_{t-1}, \ldots, X_{t-P}, \tilde{a}_t, Y_t, Y_{t-1}, \ldots,
	Y_{t-P}\}
\end{equation}

\subsection{Actions}
The agent, or trading system, only specifies the portfolio weights $a_t = (a_t^0,
\ldots, a_t^I)^T$ and therefore determines the allocation for the time interval
$[t, t+1)$. If we assume that the agent invests all of his capital at each time
step and that short-selling is not allowed, then the portfolio weights should
satisfy, for every $t \in \{0, 1, \ldots\}$, the following constraints
\begin{equation}
	\begin{cases}
		a_t^i \geq 0 \;,\; \forall i \in \{0, \ldots, I\}\\
		\sum^{I}_{i=0} a_t^i = 1 
	\end{cases}
\end{equation}
This constraint can be easily enforced by considering a parametric softmax 
policy. If short-selling is allowed, weights might also be negative. A simple approach is to assume that, for every $t \in \{0, 1, \ldots\}$, the weights satisfy
\begin{equation}
	\begin{cases}
		a_t^i \in \R \;,\; \forall i \in \{1, \ldots, I\}\\
		a_t^0 = 1 - \sum^{I}_{i=1} a_t^i
	\end{cases}
\end{equation}
Since $a_t^0$ is uniquely determined by the other weights, it is enough to
define a policy that specifies the allocation in the risky assets, e.g.\ a
Gaussian policy in $\R^I$. The obvious shortcoming of this approach is that
the agent might enter in huge short positions, which is not realistic. A first
observation is that, if the stochastic policy is concentrated around the
origin, huge long or short positions would have very small probabilities of 
being selected. Moreover, we notice that the agent would pay large fees for 
entering into large short positions, independently of the trading profits. 
Therefore, we expect the agent to learn that short positions are very expensive 
and would therefore try to avoid them.\\
Working in a continuous action space is computationally difficult and only few
reinforcement learning algorithm are well-suited to this setting, e.g. policy
gradient methods. A simpler approach is to reduce the action space to a discrete 
space. For instance, in the two assets scenario we might assume that $a_t \in
\{-1, 0, +1\}$. Thus the agent may be long ($+1$), neutral ($0$) or short
($-1$) on the risky-asset. Working in a discrete action space is more simple,
and standard value-based approaches might also be employed. 
