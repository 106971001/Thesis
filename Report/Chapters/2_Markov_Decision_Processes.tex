\chapter{Markov Decision Processes}

We consider sequential decision-making tasks that can be formulated as a
reinforcement learning (RL) problem. In RL, an agent interacts with a dynamic,
stochastic, and incompleteley known environment, with the goal of optimizing
some measure of its long-term performance. The learning problem can be 
formalized using Markov decision processes and stochastic optimal control 
theory in discrete time.

\section{Markov Decision Processes}
\begin{definition}[Markov Decision Process]
	A Markov decision process (MDP) is a tuple $<\S, \A, \calP, \calR,
	\gamma>$, where
	\begin{enumerate}[label={\roman*)}]
		\item $(\S, \calS)$ is a measurable state space.
		\item $(\A, \calA)$ is a measurable action space. 
		\item $\calP: \S \times \A \times \calS \to \R$ is a Markov transition
			kernel, i.e.
			\begin{enumerate}[label={\alph*)}]
				\item for every $s\in\S$ and $a\in\A$, $B \mapsto \calP(s,a,B)$
					  is a probability distribution over $(\S, \calS)$.
				\item for every $B\in\calS$, $(s,a) \mapsto \calP(s,a,B)$ is
					  a measurable function on $\S \times \A$.
			\end{enumerate}
		\item $\calR: \S \times \A \to \R$ is a reward function.
		\item $\gamma \in (0,1)$ is a discount factor.
   %     \item $\mu: \calS \to \R$ is a probability distribution over $(\S,
			%\calS)$.
	\end{enumerate}
\end{definition}
%$\mu$ is the probability distribution of the initial state of the system. 
The kernel $\calP$ describes the random evolution of the system: suppose that
at time $t$ the system is in state $s_t$ and that the agent takes action $a_t$,
then, regardless of the previous history of the system, the probability to find
the system in a state belonging to $B\in\calS$ at time $t+1$ is given by
$\calP(s_t, a_t, B)$, i.e.
\begin{equation}
	\calP(s_t, a_t, B) = \P{S_{t+1} \in B | S_t = s_t, A_t = a_t}
\end{equation}
Following this random transition, the agent receives a stochastic reward
$R_{t+1}$. The reward function $\calR(s_t, a_t)$ gives the expected reward
obtained when action $a_t$ is taken in state $s_t$, i.e. 
\begin{equation}
	\calR(s_t, a_t) = \E{R_{t+1} | S_t = s_t, A_t = a_t}
\end{equation}

\section{Optimal Control Problem}
At any time step, the agent selects his actions according to a certain policy. 
\begin{definition}[Policy]
	A policy is a function $\pi: \S \times \calA \to \R$ such that
	\begin{enumerate}[label={\roman*)}]
		\item for every $s \in \S$, $C \mapsto \pi(s,C)$ is a probability
			  distribution over $(\A, \calA)$. 
		\item for every $C \in \calA$, $s \mapsto \pi(s, C)$ is a measurable
			  function. 
	\end{enumerate}
\end{definition}
Intuitively, a policy represents a stochastic mapping from the current state of
the system to actions. Deterministic policies are a particular case of this 
general definition. We assumed that the agent's policy is stationary and only
depends on the current state of the system. We might in fact  consider more 
general policies that depends on the whole history of the system. However, as
we will see, we can always find an optimal policy that depends only on the
current state, so that our definition is not restrictive. A policy $\pi$ and an
initial state $s_0 \in \S$ together determine a random state-action-reward 
sequence ${\{(S_t, A_t, R_{t+1})\}}_{t\geq 0}$ with values on $\S \times \A
\times \R$ following the mechanism described above. 
\begin{definition}[History]
	Given an initial state $s_0 \in \S$ and a policy $\pi$, a history (or
	equivalently trajectory or roll-out) of the system is a random sequence
	$H_\pi = {\{(S_t, A_t)\}}_{t\geq 0}$ with values on $\S \times \A$, defined
	on some probability space $(\Omega, \mathcal{F}, \mathbb{P})$, such that for 
	$t = 0, 1, \ldots$
	\begin{equation}
		\begin{cases}
			S_0 = s_0\\
			A_t \sim \pi(S_t, \cdot)\\
			S_{t+1} \sim \calP(S_t, A_t, \cdot)\\
		\end{cases}
	\end{equation}
	we will denote by $(\H, \calH)$ the measurable space of all possible
	histories. 
\end{definition}
Moreover, we observe that
\begin{enumerate}[label={\roman*)}]
	\item the state sequence ${\{S_t\}}_{t\geq 0}$ is a Markov process $<\S,
		  \calP_\pi>$
	  \item the state-reward sequence ${\{(S_t, R_t)\}}_{t\geq 0}$ is a Markov 
		  reward process $<\S, \calP_\pi, \calR_\pi, \gamma>$
\end{enumerate}
where we denoted 
\begin{equation}
	\begin{split}
		\calP_\pi(s, s') &= \int_\A \pi(s, a) \calP(s, a, s') da\\
		\calR_\pi(s) &= \int_\A \pi(s, a) \calR(s, a) da\\
	\end{split}
\end{equation}
The goal of the agent is to maximize his expected return. 
\begin{definition}[Return]
	The return is the total discounted reward obtained by the agent starting 
	from $t$  
	\begin{equation}
		G_t = \sum^{\infty}_{t=0} \gamma^t R_{t+k+1} 
	\end{equation}
\end{definition}
where $0 < \gamma < 1$ is the discount factor. The discount factor models the
trade-off between immediate and delayed reward: if $\gamma = 0$ the agent
selects his actions in a myopic way, while if $\gamma \to 1$ he acts in a
far-sighted manner. There are other possilble reasons for discounting future
rewards. The first is because it is mathematically convenient, as it avoids
infinite returns and it solves many convergence issues. Another interpretation
is that it models the uncertainty about the future, which may not be fully
represented. The discount factor can also be seen as the probability that the
world does not stop at a given time step. Finally, the financial interpration
is that discounting gives the present value of future rewards. Since the return
are stochastic, we consider their expected value. 
\begin{definition}[State-Value Function]
	The state-value function $V_\pi: \S \to \R$ is the expected return that can
	be obtained starting from a state and following policy $\pi$
	\begin{equation}
		V_\pi(s) = \E[\pi]{G_t|S_t = s}
	\end{equation}
\end{definition}
where $\mathbb{E}_{\pi}$ indicates that all the actions are selected according
to policy $\pi$. In reinforcement learning, it is useful to consider another
function 
\begin{definition}[Action-Value Function]
	The action-value function $Q_\pi: \S \times \A \to \R$ is the expected 
	return that can be obtained starting from a state, taking an action and
	then following policy $\pi$
	\begin{equation}
		Q_\pi(s,a) = \E[\pi]{G_t|S_t = s, A_t = a}
	\end{equation}
\end{definition}
\begin{definition}[Optimal State-Value Function]
	The optimal state-value function $V_*: \S \to \R$ is the largest expected 
	return that can be obtained starting from a state
	\begin{equation}
		V_*(s) = \sup_\pi V_\pi(s)
	\end{equation}
\end{definition}
\begin{definition}[Optimal Action-Value Function]
	The optimal action-value function $Q_*: \S \times \A \to \R$ is the largest
	expected return that can be obtained starting from a state and taking an
	action
	\begin{equation}
		Q_*(s,a) = \sup_\pi Q_\pi(s,a)
	\end{equation}
\end{definition}
The agent's goal is to select a policy $\pi_*$ that maximize his expected return
in all possible states. Such a policy is called \emph{optimal}. More formally,
we introduce the following partial ordering in the policy space
\begin{equation}
	\pi \succeq \pi' \Leftrightarrow V_\pi(s) \geq V_{\pi'}(s) \;\;\; \forall s \in \S
\end{equation}
Then the optimal policy $\pi_* \succeq \pi$, $\forall \pi$.

\section{Bellman Equations}



\section{Risk-Sensitive MDP}



\section{Average Reward Formulation}



