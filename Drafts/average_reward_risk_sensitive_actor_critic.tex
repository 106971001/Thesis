\documentclass[a4paper,11pt]{article}

\usepackage{Style/thesis}

\begin{document}

\title{Average Reward Risk-Sensitive Actor-Critic Algorithm}
\author{Pierpaolo Necchi}
\maketitle

\section{Average Reward}
Most of the research in RL has studied a problem formulation where agents maxi-
mize the cumulative sum of rewards. However, this approach cannot handle 
infinite horizon tasks, where there are no absorbing goal states, without
discounting future rewards. Traditionally, discounting has served two purposes.
In some domains, such as economics, discounting can be used to represent
interest earned on rewards, so that an action that generates an immediate
reward will be preferred over one that generates the same reward some steps
into the future. However, the typical domains studied in RL, such as robotics
or games, do not fall in this category. In fact, many RL tasks have absorbing
goal states, where the aim of the agent is to get to a given goal state as
quickly as possible. Such tasks can be solved using undiscounted RL methods.  
Clearly, discounting is only really necessary in cyclical tasks, where the 
cumulative reward sum can be unbounded. More natural long-term measure of
optimality exists for such cyclical tasks, based on maximizing the average
reward per action \cite{mahadevan1996average}. 

\section{Average Reward Control Problem}
In the average reward setting, the goal of the agent is to find a policy that
maximizes the expected reward per step. 
\begin{definition}[Average Reward]
	The average reward $\rho_\pi$ associated to a policy $\pi$ is defined as  
	\begin{equation}
		\begin{split}
			\rho_\pi &= \lim_{T\to\infty} \frac{1}{T} \E{ \sum^{T-1}_{t=0} R_{t+1}}\\
					 &= \int_\S d_\pi(s) \int_\A \pi(s, a) \calR(s,a) da ds\\
					 &= \E[\substack{S\sim d_\pi \\ A \sim \pi(S,
						\cdot)}]{\calR(S,A)}
		\end{split}
	\end{equation}
where $d_\pi$ is the stationary distribution induced by policy $\pi$.
\end{definition}
In the risk-neutral setting, the optimal policy solves the following
optimization problem  
\begin{equation}
	\rho_* = \sup_\pi \rho(\pi)
\end{equation}
In this setting, we introduce the \emph{average adjusted} value and action-value 
functions. 
\begin{definition}[Average Adjusted State-Value Function]
	The average adjusted state-value function $V_\pi : \S \to \R$ is the
	expected residual return that can be obtained starting from a state and
	following policy $\pi$
	\begin{equation}
		V_\pi(s) = \E[\pi]{\sum^{\infty}_{t=0} \left(R_{t+1} - \rho_\pi\right)
		\bigg| S_0 = s}
	\end{equation}
\end{definition}
\begin{definition}[Average Adjusted Action-Value Function]
	The average adjusted action-value function $Q_\pi : \S \times \A \to \R$ is 
	the expected residual return that can be obtained starting from a state,
	taking an action and then following policy $\pi$
	\begin{equation}
		Q_\pi(s, a) = \E[\pi]{\sum^{\infty}_{t=0} \left(R_{t+1} -
			\rho_\pi\right) \bigg| S_0 = s, A_0
		= a}
	\end{equation}
\end{definition}
We have the following relation between the state-value function and the
action-value function
\begin{equation}
	V_\pi(s) = \int_\A \pi(s,a) Q_\pi(s,a)
\end{equation}
These functions satisfy the following Bellman equations
\begin{equation}
	\begin{split}
		\rho_\pi + V_\pi(s) &= \calR_\pi(s) + T_\pi V_\pi(s)\\
		\rho_\pi + Q_\pi(s,a) &= \calR(s,a) + T_a V_\pi(s)\\
	\end{split}
\end{equation}
where $T_a$ denotes the transition operator for action $a$ while $T_\pi$
denotes the transition operator for the policy $\pi$, i.e. 
\begin{equation}
	\begin{split}
		T_a V(s) &= \int_\S \calP(s, a, s') V_\pi(s')\\
		T_\pi V(s) &= \int_\A \pi(s,a) \int_\S \calP(s,a,s') V_\pi(s')\\ 
	\end{split}
\end{equation}

\section{Risk-Sensitive Average Reward Control Problem}
In many application, in addition to maximining the average reward, the agent
may want to control risk by minimizing some measure of viariability in rewards.
In \cite{prashanth2014actor}, the authors consider the long-run variance of
$\pi$, defined as 
\begin{definition}[Long-Run Variance]
	Given a policy $\pi$, the long-run variance $\Lambda_\pi$ is 
	\begin{equation}
		\begin{split}
			\Lambda_\pi &= \lim_{T \to \infty} \frac{1}{T} \E[\pi]{
			\sum^{T-1}_{t=0} (R_{t+1} - \rho_\pi)^2}\\
			&= \int_\A d_\pi(s) \int_\A \pi(s,a) \left(\calR(s,a) - \rho_\pi
			\right)^2 da ds\\
			&= \E[\substack{S\sim d_\pi \\ A \sim \pi(S, \cdot)}]{(\calR(S,A) -
				\rho_\pi)^2}
		\end{split}
	\end{equation}
\end{definition}
We can split the long-run variance in the following way
\begin{equation}
	\Lambda_\pi = \eta_\pi - \rho_\pi^2 
\end{equation}
where $\eta_\pi$ is the expected square reward per step  
\begin{equation}
	\begin{split}
		\eta_\pi &= \int_\S d_\pi(s) \int_\A \pi(s,a) \calR(s,a)^2\\
				 &= \E[\substack{S\sim d_\pi \\ A \sim \pi(S, \cdot)}]{\calR(S,A)^2}	
	\end{split}
\end{equation}
We introduce the state-value and action-value functions associated with the
square reward under policy $\pi$
\begin{definition}[Average Adjusted Square State-Value Function]
	The average adjusted square state-value function $U_\pi : \S \to \R$ is the
	expected square residual return that can be obtained starting from a state 
	and following policy $\pi$
	\begin{equation}
		U_\pi(s) = \E[\pi]{\sum^{\infty}_{t=0} \left(R_{t+1}^2 - \eta_\pi\right)
		\bigg| S_0 = s}
	\end{equation}
\end{definition}
\begin{definition}[Average Adjusted Square Action-Value Function]
	The average adjusted square action-value function $Q_\pi : \S \times \A \to 
	\R$ is the expected residual square return that can be obtained starting from a
	state, taking an action and then following policy $\pi$
	\begin{equation}
		W_\pi(s, a) = \E[\pi]{\sum^{\infty}_{t=0} \left(R_{t+1}^2 -
			\eta_\pi\right) \bigg| S_0 = s, A_0 = a}
	\end{equation}
\end{definition}
The average adjusted square action-value function satisfies the following 
Bellman equation
\begin{equation}
	\eta_\pi + W_\pi(s,a) = \calR(s,a)^2 + \int_\S \calP(s,a,s') U_\pi(s') ds'\\
\end{equation}
Moreover, the average adjusted square state-value function is related to the
action-value function by the following
\begin{equation}
	U_\pi(s) = \int_\A \pi(s,a) W_\pi(s,a)
\end{equation}
Using this equality, we can write an analogous Bellman equation for the
square state-value function. 
In the risk-sensitive setting, the agent wants to find a policy that solves the
following optimization problem 
\begin{equation}\label{eq:risk_sensitive_problem}
	\begin{cases}
		\max_\pi \rho_\pi\\
		\text{subject to}\ \Lambda_\pi \leq \alpha\\
	\end{cases}
\end{equation}
for a given $\alpha > 0$. Using the Lagrangian relaxation procedure, we can 
recast (\ref{eq:risk_sensitive_problem}) to the following uncostrained problem
\begin{equation}
	\max_\lambda \min_\pi L(\theta, \lambda) = - \rho_\pi + \lambda 
	(\Lambda_\theta - \alpha)
\end{equation}
Let us observe that the discussion can be easily extended to other
risk-sensitive performance measures, such as the standard mean-variance
criterion or the Sharpe ratio. 

\section{Risk-Sensitive Policy Gradient}

\section{Risk-Sensitive Actor-Critic Algorithm}

% Bibliography
\bibliographystyle{siam}
\bibliography{Bibliography/bibliography}

\end{document}
