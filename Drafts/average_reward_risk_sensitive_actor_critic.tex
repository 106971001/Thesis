\documentclass[a4paper,11pt]{article}

\usepackage{Style/thesis}

\begin{document}

\title{Average Reward Risk-Sensitive Actor-Critic Algorithm}
\author{Pierpaolo Necchi}
\maketitle

\section{Average Reward}
Most of the research in RL has studied a problem formulation where agents maxi-
mize the cumulative sum of rewards. However, this approach cannot handle 
infinite horizon tasks, where there are no absorbing goal states, without
discounting future rewards. Traditionally, discounting has served two purposes.
In some domains, such as economics, discounting can be used to represent
interest earned on rewards, so that an action that generates an immediate
reward will be preferred over one that generates the same reward some steps
into the future. However, the typical domains studied in RL, such as robotics
or games, do not fall in this category. In fact, many RL tasks have absorbing
goal states, where the aim of the agent is to get to a given goal state as
quickly as possible. Such tasks can be solved using undiscounted RL methods.  
Clearly, discounting is only really necessary in cyclical tasks, where the 
cumulative reward sum can be unbounded. More natural long-term measure of
optimality exists for such cyclical tasks, based on maximizing the average
reward per action \cite{mahadevan1996average}. 

\section{Average Reward Control Problem}
In the average reward setting, the goal of the agent is to find a policy that
maximizes the expected reward per step. 
\begin{definition}[Average Reward]
	The average reward $\rho_\pi$ associated to a policy $\pi$ is defined as  
	\begin{equation}
		\begin{split}
			\rho_\pi &= \lim_{T\to\infty} \frac{1}{T} \E[\pi]{ \sum^{T-1}_{t=0} R_{t+1}}\\
					 &= \E[\substack{S\sim d_\pi \\ A \sim \pi}]{\calR(S,A)}\\ 
					 &= \int_\S d_\pi(s) \int_\A \pi(s, a) \calR(s,a) da ds\\ 
		\end{split}
	\end{equation}
where $d_\pi$ is the stationary distribution of the Markov process induced by 
policy $\pi$.
\end{definition}
In the risk-neutral setting, the agent aims to find an \emph{average optimal}
policy
\begin{equation}
	\pi_* = \argsup_\pi \rho_\pi
\end{equation}
In this setting, we introduce the \emph{average adjusted} value and action-value 
functions. 
\begin{definition}[Average Adjusted State-Value Function]
	The average adjusted state-value function $V_\pi : \S \to \R$ is the
	expected residual return that can be obtained starting from a state and
	following policy $\pi$
	\begin{equation}
		V_\pi(s) = \E[\pi]{\sum^{\infty}_{t=0} \left(R_{t+1} - \rho_\pi\right)
		\bigg| S_0 = s}
	\end{equation}
\end{definition}
The term $V_\pi(s)$ is usually referred to as the \emph{bias} value, or the
\emph{relative} value, since it represents the relative difference in total
reward gained starting from a state $s$ as opposed to a generic state. 
$\rho_\pi$ serves as a baseline that allows to avoid divergence in the value
function definition.
\begin{definition}[Average Adjusted Action-Value Function]
	The average adjusted action-value function $Q_\pi : \S \times \A \to \R$ is 
	the expected residual return that can be obtained starting from a state,
	taking an action and then following policy $\pi$
	\begin{equation}
		Q_\pi(s, a) = \E[\pi]{\sum^{\infty}_{t=0} \left(R_{t+1} -
			\rho_\pi\right) \bigg| S_0 = s, A_0
		= a}
	\end{equation}
\end{definition}
We have the following relation between the state-value function and the
action-value function
\begin{equation}\label{eq:VQ_equality}
	V_\pi(s) = \int_\A \pi(s,a) Q_\pi(s,a)
\end{equation}
The action-value function satisfies the following Bellman equation 
\begin{equation}
	\rho_\pi + Q_\pi(s,a) = \calR(s,a) + \int_\S \calP(s,a,s') V_\pi(s') ds'
\end{equation}
Using equality (\ref{eq:VQ_equality}), we obtain the Bellman equation for the
state-value function
\begin{equation}
	\rho_\pi + V_\pi(s) = \int_\A \pi(s,a) \left[\calR(s,a) + \int_\S 
	\calP(s,a,s') V_\pi(s') ds'\right] da
\end{equation}
Denoting by $T_a$ (resp. $T_\pi$) the transition operator for action $a$ (resp. 
for policy $\pi$)
\begin{equation}
	\begin{split}
		T_a V(s) &= \int_\S \calP(s, a, s') V_\pi(s')\\
		T_\pi V(s) &= \int_\A \pi(s,a) \int_\S \calP(s,a,s') V_\pi(s')\\ 
	\end{split}
\end{equation}
The Bellman equations can be rewritten in the shorter form
\begin{equation}
	\begin{split}
		\rho_\pi + V_\pi(s) &= \calR_\pi(s) + T_\pi V_\pi(s)\\
		\rho_\pi + Q_\pi(s,a) &= \calR(s,a) + T_a V_\pi(s)\\
	\end{split}
\end{equation}
In the discrete case, where the transition operator correspond to matrices, 
these Bellman equations become linear systems that can be solved to obtain the
value functions. 

\section{Risk-Sensitive Average Reward Control Problem}
In many application, in addition to maximining the average reward, the agent
may want to control risk by minimizing some measure of viariability in rewards.
In \cite{prashanth2014actor}, the authors consider the long-run variance of
$\pi$, defined as 
\begin{definition}[Long-Run Variance]
	The long-run variance $\Lambda_\pi$ under policy $\pi$ 
	\begin{equation}
		\begin{split}
			\Lambda_\pi &= \lim_{T \to \infty} \frac{1}{T} \E[\pi]{
			\sum^{T-1}_{t=0} (R_{t+1} - \rho_\pi)^2}\\
			&= \E[\substack{S\sim d_\pi \\ A \sim \pi}]{(\calR(S,A) -
				\rho_\pi)^2}\\
			&= \int_\A d_\pi(s) \int_\A \pi(s,a) \left(\calR(s,a) - \rho_\pi
			\right)^2 da ds\\
		\end{split}
	\end{equation}
\end{definition}
The long-run variance can be decomposed as follows
\begin{equation}
	\Lambda_\pi = \eta_\pi - \rho_\pi^2 
\end{equation}
where $\eta_\pi$ is the average square reward per step  
\begin{definition}[Average Square Reward]
	\begin{equation}
		\begin{split}
			\eta_\pi &= \E[\substack{S\sim d_\pi \\ A \sim \pi}]{\calR(S,A)^2}\\	
					 &= \int_\S d_\pi(s) \int_\A \pi(s,a) \calR(s,a)^2\\
		\end{split}
	\end{equation}
\end{definition}
As before, we introduce the residual state-value and action-value functions 
associated with the square reward under policy $\pi$
\begin{definition}[Average Adjusted Square State-Value Function]
	The average adjusted square state-value function $U_\pi : \S \to \R$ is the
	expected square residual return that can be obtained starting from a state 
	and following policy $\pi$
	\begin{equation}
		U_\pi(s) = \E[\pi]{\sum^{\infty}_{t=0} \left(R_{t+1}^2 - \eta_\pi\right)
		\bigg| S_0 = s}
	\end{equation}
\end{definition}
\begin{definition}[Average Adjusted Square Action-Value Function]
	The average adjusted square action-value function $Q_\pi : \S \times \A \to 
	\R$ is the expected residual square return that can be obtained starting from a
	state, taking an action and then following policy $\pi$
	\begin{equation}
		W_\pi(s, a) = \E[\pi]{\sum^{\infty}_{t=0} \left(R_{t+1}^2 -
			\eta_\pi\right) \bigg| S_0 = s, A_0 = a}
	\end{equation}
\end{definition}
The average adjusted square action-value function satisfies the following 
Bellman equation
\begin{equation}
	\eta_\pi + W_\pi(s,a) = \calR(s,a)^2 + \int_\S \calP(s,a,s') U_\pi(s') ds'\\
\end{equation}
Moreover, the average adjusted square state-value function is related to the
action-value function by the following
\begin{equation}
	U_\pi(s) = \int_\A \pi(s,a) W_\pi(s,a)
\end{equation}
Using this equality, we can write an analogous Bellman equation for the
square state-value function. 
In the risk-sensitive setting, the agent wants to find a policy that solves the
following optimization problem 
\begin{equation}\label{eq:risk_sensitive_problem}
	\begin{cases}
		\max_\pi \rho_\pi\\
		\text{subject to}\ \Lambda_\pi \leq \alpha\\
	\end{cases}
\end{equation}
for a given $\alpha > 0$. Using the Lagrangian relaxation procedure, we can 
recast (\ref{eq:risk_sensitive_problem}) to the following uncostrained problem
\begin{equation}
	\max_\lambda \min_\pi L(\pi, \lambda) = - \rho_\pi + \lambda 
	(\Lambda_\theta - \alpha)
\end{equation}
Let us observe that the discussion can be easily extended to other
risk-sensitive performance measures, such as the standard mean-variance
criterion or the Sharpe ratio. 

\section{Risk-Sensitive Policy Gradient}
Let us consider a family of parametrized policies $\pi_\theta$, with $\theta
\in \Theta \subseteq \R^{D_\theta}$. The optimization problem then becomes
\begin{equation}
	\max_\lambda \min_\theta L(\theta, \lambda) = - \rho(\theta) + \lambda
	(\Lambda(\theta) - \alpha)
\end{equation}
Using a policy gradient approach, the policy parameters are updated following 
the gradient ascent direction
\begin{equation}
	\nabla_\theta L(\theta, \lambda) = - \nabla_\theta \rho(\theta) + \lambda
	\nabla_\theta \Lambda(\theta)
\end{equation}
while the Lagrange multiplier is updated following the gradient descent
direction  
\begin{equation}
	\nabla_\lambda L(\theta, \lambda) = \Lambda(\theta) - \alpha
\end{equation}
Since 
\begin{equation}
	\nabla_\theta \Lambda(\theta) = \nabla_\theta \eta(\theta) - 2 \rho(\theta)
	\nabla_\theta \rho(\theta)
\end{equation}
it is enough to compute $\nabla_\theta \eta(\theta)$ and $\nabla_\theta
\rho(\theta)$. These quantities are provided by the policy gradient theorem 
\begin{theorem}[Policy Gradient]
	\begin{equation}\label{eq:policy_gradient_theorem_Q}
		\begin{split}
			\nabla_\theta \rho(\theta) &= \E[\substack{S\sim d_\pi \\ 
			A \sim \pi}]{\nabla_\theta \log \pi_\theta(S,A) Q_\theta(S,A)}\\
			&= \int_\S d_\pi(s) \int_\A \pi(s,a) \nabla_\theta \log
			\pi_\theta(s,a) Q_\theta(s,a) da ds	
		\end{split}
	\end{equation}
	\begin{equation}\label{eq:policy_gradient_theorem_W}
		\begin{split}
			\nabla_\theta \eta(\theta) &= \E[\substack{S\sim d_\pi \\ 
			A \sim \pi}]{\nabla_\theta \log \pi_\theta(S,A) W_\theta(S,A)}\\
			&= \int_\S d_\pi(s) \int_\A \pi(s,a) \nabla_\theta \log
			\pi_\theta(s,a) W_\theta(s,a) da ds	
		\end{split}
	\end{equation}
\end{theorem}
\begin{proof}
	Eq. (\ref{eq:policy_gradient_theorem_Q}) is the standard policy gradient
	and its proof can be found in the literature, e.g. \cite{sutton1999policy}.
	Eq. (\ref{eq:policy_gradient_theorem_W}) can be shown in a similar fashion.
	TODO
\end{proof}
As in the standard risk-neutral case, a state-dependent baseline can be 
introduced in both gradients without changing the result. In particular, by 
using the average adjusted value functions as baseline, we can replace the 
average adjusted action-value functions with the following advantage functions
\begin{equation}
	A_\theta(s,a) = Q_\theta(s,a) - V_\theta(s)
\end{equation}
\begin{equation}
	B_\theta(s,a) = W_\theta(s,a) - U_\theta(s)
\end{equation}
Intuitively, the advantage functions measure how good is to take action $a$ in
state $s$ compared to simply following the policy $\pi_\theta$. The use of a
baseline also serves the purpose of reducing the variance of the gradient
estimates. Thus, the gradients can be written as 
\begin{equation}\label{eq:pg_advantage_Q}
	\nabla_\theta \rho(\theta) = \E[\substack{S\sim d_\pi \\ 
A \sim \pi}]{\nabla_\theta \log \pi_\theta(S,A) A_\theta(S,A)}
\end{equation}
\begin{equation}\label{eq:pg_advantage_W}
	\nabla_\theta \eta(\theta) = \E[\substack{S\sim d_\pi \\ 
A \sim \pi}]{\nabla_\theta \log \pi_\theta(S,A) B_\theta(S,A)}
\end{equation}
Results of this type form the basis of many actor-critic algorithms, which
employ an approximation of the advantage function to obtain a more accurate
estimate of the objective function. 

\section{Risk-Sensitive Actor-Critic Algorithm}
In \cite{prashanth2014actor}, starting from (\ref{eq:pg_advantage_Q}) and 
(\ref{eq:pg_advantage_W}), the authors derive a risk-sensitive actor-critic 
algorithm for the average reward setting. In the algorithm proposed, the
advantage functions are approximated using a temporal difference (TD) scheme in
which a critic maintains a linear approximation of the value functions. More in
detail, let $\delta_n^A$ and $\delta_n^B$ be the TD errors for residual value
and square value functions
\begin{equation}
	\begin{split}
		\delta_t^A &= R_{t+1} - \widehat{\rho}_{t+1} + \widehat{V}(S_{t+1}) -
		\widehat{V}(S_t)\\
		\delta_t^B &= R_{t+1}^2 - \widehat{\eta}_{t+1} + \widehat{U}(S_{t+1}) -
		\widehat{U}(S_t)\\
	\end{split}
\end{equation}
where $\widehat{V}$, $\widehat{U}$, $\widehat{\rho}$ and $\widehat{\eta}$ are
unbiased estimate of $V_\theta$, $U_\theta$, $\rho(\theta)$ and $\eta(\theta)$
respectively. It is easy to show that $\delta_t^A$ and $\delta_t^B$ are
unbiased estimates of the advantage functions.
\begin{proposition}[Temporal Difference Errors]
	\begin{equation}
		\begin{split}
			\E[\theta]{\delta_t^A|S_t = s, A_t = a} &= A_\theta(s, a)\\
			\E[\theta]{\delta_t^B|S_t = s, A_t = a} &= B_\theta(s, a)\\
		\end{split}
	\end{equation}
\end{proposition}
\begin{proof}
	TODO
\end{proof}
Denoting by $\psi_t = \psi(S_t, A_t) = \nabla_\theta \log \pi(S_t, A_t)$ the
compatible feature \cite{sutton1999policy}, we can easily obtain an unbiased
estimate of the gradients 
\begin{equation}
	\begin{split}
		\nabla_\theta \rho(\theta) &\approx \psi_t \delta_t^A\\
		\nabla_\theta \eta(\theta) &\approx \psi_t \delta_t^B\\
	\end{split}
\end{equation}
The value functions are linearly approximated using some veatures vectors
$\Phi_V: \S \to \R^{D_V}$ and $\Phi_U: \S \to \R^{D_U}$ as follows
\begin{equation}
	\begin{split}
		\widehat{V}(s) &= v^T \Phi_V(s)\\
		\widehat{U}(s) &= u^T \Phi_U(s)\\
	\end{split}
\end{equation}
Given all these ingredients, the authors propose a three time-scale stochastic
approximation algorithm whose pseudo-code is illustrated in Algorithm
%\ref{algo:average_reward_risk_sensitive_actor_critic_algorithm}.

% \begin{algorithm}
	%\caption{Average Reward Risk-Sensitive Actor-Critic Algorithm}
	% \label{algo:average_reward_risk_sensitive_actor_critic_algorithm}
	%\begin{algorithmic}
	%	test		
	%\end{algorithmic}
% \end{algorithm}





% Bibliography
\bibliographystyle{siam}
\bibliography{Bibliography/bibliography}

\end{document}
